\documentclass{article}
\usepackage{Sweave}
\begin{document}
\input{report-concordance}

\title{Unemployment and its Educational Predictors}
\author{
Alcorn, Bryan\\
Shane, Joelle\\
Vogel, Abby\\
Vogel, Todd\\
}
\date{\today}

\maketitle

\begin{abstract}
In this research we look into the efficacy of secondary education on post graduation unemployment as well as whether public spending is effective at improving the quality of education.  To do this we utilized 3 predictive models (ridge, lasso, and pcr) and hypothesis testing.
\end{abstract}

\end{document}\documentclass{article}
\usepackage{Sweave}
\begin{document}


\title{The Title}
\author{A. Author}
\date{\today}

\maketitle

\section{Introduction}

According to the US Bureau of Labor Statistics, those with a Bachelor's Degree have nearly twice the weekly earnings of those who have only a High School Degree (2006 Data). With the cost of education rising, it is ever important to be sure that the college a student choses attend will equipt them to have a career that can pay back their student loans. 

The College Scorecard ia a source of reliable nationwide data on over 7,000 institutions. This site, collegescorecard.gov, is a project of the Department of Education to aggregate annual data on higher education institutions. It has information from 19 years assembled from numerous sources. More information on the data can be found at collegescorecard.gov/data. 

The goal of this data was to create a resource for students, parents, and counsellors to be able to access information about the preformance of colleges and universities, and compare which school is the best fit. President Obama announced the New College Scorecard in September of 2015 as a means for people to better find an affordable and reputable institution to prepare themselves to enter the workforce. 

The goal of this analysis is to determine the features of institutions that lead to high employment rates post-graduation in order to answer the research question: does public spending improve the quality of education? We assume that employment after graduation is an accurate reflection of the quality of education the students recieve at a particular school. By focusing on public spending through data on financial aid reciepients, in conjunciton with other predictor variables, this analysis will reveal the ways in which public spending can impact the quality of a post-secondary schooling. This analysis could also be used by schools that wish to increase their post-graduation employment rates.

\end{document}\documentclass{article}
\usepackage{Sweave}
\begin{document}


\title{The Title}
\author{A. Author}
\date{\today}

\maketitle

\section{Data}

This analysis uses data from 2006. The 2006 data was selected because it contains the unemployment rate via Census Data, which was selected to be the response variable. Unemplyment rate was chosen as the response variable because it is a robust measure across different sized and located institutions.

To clean the data, all factor variables were changed to numeric. Each variable with more than 50\% null values was removed. The remaining missing values were replaced by the median of the variable. Data was both mean-centered and standardized.

Two data sets were created for the modeling. First, the full cleaned data with over 500 variables was used. Initially, Principle Component Analysis was used to reduce the number of variables, but resulted in all variables in the first component. Because of this, the full cleaned data was used in the model.

In addition, a reduced set of predictor variables were selected from the full data by their perceived importance in educational success. This shortened data set was pulled from the data before the greater than 50\% NA variables were removed. 

The shortened data includes:

\begin{itemize}
\item Unemployment Rate (as response variable)
\item Instructional expenditures per full-time equivalent student
\item In-state tuition and fees
\item Average faculty salary
\item Completion rate for first-time, full-time students at four-year institutions (150\% of expected time to completion/6 years)
\item Completion rate for first-time, full-time students at less-than-four-year institutions (150\% of expected time to completion)
\item First-time, full-time student retention rate at four-year institutions
\item Percent of students who received a Pell Grant at the institution and who completed in 2 years at original institution
\item Percent of students who received a Pell Grant at the institution and who completed in 3 years at original institution
\item Percent of students who received a Pell Grant at the institution and who completed in 4 years at original institution
\item Two-year cohort default rate
\end{itemize}

Other variables were considered, but many weren't available in the 2006 data. 

Further analysis across years would provide a better look into weather or not a change in spending impacts the post-graduation unemployment rate, but was beyond the scope of the data available via The College Scorecard. 

\end{document}
\documentclass{article}
\usepackage{Sweave}
\begin{document}


\title{The Title}
\author{A. Author}
\date{\today}

\maketitle



\section{Methods}

This project involved the use and implementation of 3 different linear models as well as hypothesis testing.

\subsection{Ridge Regression}

Ridge Regression is the first of the two shrinkage modelling methods we used.  When using shrinkage methods the goal is to penalize certain parameters that should have a less significant effect on the model. To do so we use the tuning parameter, $\lambda$, times $\sum_{j=1}^{p}\beta_{j}^{2}$ to yield the shrinkage penalty. We then determine the coefficients that minimize the following equation:

\begin{equation}
\sum_{i=1}^{n}(y_i-\beta_0-\sum_{j=1}^{p}\beta_jx_{ij}) + \lambda\sum_{j=1}^{p}\beta_{j}^{2}
\end{equation}

To find these coefficients we used the function $cv.glmnet()$ to determine through cross validation which $\lambda$ value minimizes the above function (when $\lambda$ was a given sequence of numbers under $grid$, and $alpha$ was set to 0). From there we calculated the mean squared error (a measure of predicted power), by using the function:

\begin{equation}
MSE = \frac{1}{n}\sum_{i=1}^{n}(y_i - y_{di})^2
\end{equation}

\subsection{Lasso Regression}

Lasso regression is the second and final shrinkage modelling method used in this project. Although it is very similar to ridge regression there are a few key differences: the shrinkage penalty is now $\lambda$, times $\sum_{j=1}^{p}\beta_{j}^{2}$ ($\beta$ is not squared), and lasso allows for the removal of certain variables (not just dampening their effect). To determine the coefficients we must look for the $\beta$'s that minimize:

\begin{equation}
\sum_{i=1}^{n}(y_i-\beta_0-\sum_{j=1}^{p}\beta_jx_{ij}) + \lambda\sum_{j=1}^{p}|\beta_{j}|
\end{equation}

Again, to find the coefficient we used $cv.glmnet()$ and set $\lambda$ to $grid$. However, now, $alpha$ was set to 1. Finally, we calculated MSE to determine the predictive power of our model.

\subsection{Principle Components Regression}

PCR is the first of two dimension reduction modelling methods we used.  This method labors under the assumption that a subset of all of the predictor variables account for the vast majority of the variance.  These more significant variables are referred to as principle components (M).  PCR works by setting M equal to some reduced number of variables and running cross validation on, the model with the lowest cross validation error is selected.

To develop a model through PCR we used the $pcr()$ function and set $validation$= CV.  We then found the model in which PRESS was larger to avoid overfitting. Finally, we calculated MSE to determine the predictive power of our model.

\subsection{Hypothesis Testing}

Hypothesis tests allows for comparison across groups. By separating the data points into categories based on how much funding the school received, we can then compare those categories based on unemployment rate and see how they differ.


To utilize this method we:
\begin{itemize}
\item Separated the data into groups by amount of funding the school received (low, low-mid, mid-high, high). 
\item Created hypothesis test, $W(u)$, to describe the difference being analyzed.
\item Computed the estimate for effect of funding and variance of that effect.
\end{itemize}


\end{document}\documentclass{article}
\usepackage{Sweave}
\begin{document}


\title{The Title}
\author{A. Author}
\date{\today}

\maketitle


\section{Analysis}

\subsection{Modelling}
In the Results section of the report, we determined the mean squared error values for each of our 3 models (one model with every variable from the data included and one where we preselected 10 variables).  For the full Ridge, Lasso, and PCR the MSE's were 0.186, 0.195, and 0.271 respectively. For the shor Ridge, Lasso, and PCR the MSE's were 12.797, 17.114, and 1.446 respectively. Because our data was mean centered and standardized, all of these values exist on the same scale and can thus be prepared.  Mean squared error values represent the average sum of the errors between actual and predicted response values. Therefore, the smaller the MSE value the smaller the error, and the better the predictive model.  Given this, 0.186 is our smallest MSE value meaning Ridge is our best predictive model. These results are unsurprising to an extent, each of the MSE's for the shortened models were much higher, meaning that preselecting predictors negatively effects the modelling process.  

In order to determine which variables most affected unemployment rates we look at the absolute values of the coefficients from our best model (ridge) and look at the top 5. 

% latex table generated in R 3.3.2 by xtable 1.8-2 package
% Sun Dec  4 22:07:12 2016
\begin{table}[ht]
\centering
\begin{tabular}{rlr}
  \hline
 & Predictor & Coefficient \\ 
  \hline
1 & POVERTY\_RATE & 0.72 \\ 
  2 & PCT\_WHITE & -0.42 \\ 
  3 & PCT\_BA & -0.18 \\ 
  4 & PCT\_BLACK & -0.15 \\ 
  5 & MARRIED & -0.13 \\ 
   \hline
\end{tabular}
\caption{Information about top 5  Ridge Model Coefficients} 
\end{table}
This result means that the 5 most important predictors of unemployment rate are poverty rate, \% white, \% bachelors degrees given (of total degrees given), \% black, and married.  Schools with higher poverty rates have higher unemployment after graduation, wheres the schools with higher values of the other 4 variables with have lower rates of unemployment.

\subsection{Hypothesis Testing}
The estimate for the effect of spending on quality of education recieved was about -5.4\%. This means every higher spending bracket is expected to have a 5.4\% increase in unemployment. The variance on this value was about 0.0025, making it significant. After separating the data by average faculty salary, we can see that unemployment was the highest among schools that had low average faculty salary and the lowest among schools taht had high average faculty salary. We still estimate that higher spending is associated with higher unemployment, but since controlling for just a single factor showed a decrease in the effect, we could expect continual descreases or even a reversal of the effect if we continued to control for more and more variables.


\end{document}\documentclass{article}
\usepackage{Sweave, graphicx}
\begin{document}

\title{The Title}
\author{A. Author}
\date{\today}
\maketitle




\section{Results}

\subsection{Modeling}
After forming the aforementioned regression models we found 10 coefficients that represent the best fit for each model. The resulting predictive function looks like:

\begin{equation}
Y = \beta_0 + \beta_1 X_1 + \beta_2 X_2 + \beta_3 X_3 + \beta_4 X_4 + \beta_5 X_5 + \beta_6 X_6 + \beta_7 X_7 + \beta_8 X_8 + \beta_9 X_9 + \beta_{10} X_{10}
\end{equation}

